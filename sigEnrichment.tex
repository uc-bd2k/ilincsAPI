\documentclass[]{article}
\usepackage{lmodern}
\usepackage{amssymb,amsmath}
\usepackage{ifxetex,ifluatex}
\usepackage{fixltx2e} % provides \textsubscript
\ifnum 0\ifxetex 1\fi\ifluatex 1\fi=0 % if pdftex
  \usepackage[T1]{fontenc}
  \usepackage[utf8]{inputenc}
\else % if luatex or xelatex
  \ifxetex
    \usepackage{mathspec}
  \else
    \usepackage{fontspec}
  \fi
  \defaultfontfeatures{Ligatures=TeX,Scale=MatchLowercase}
\fi
% use upquote if available, for straight quotes in verbatim environments
\IfFileExists{upquote.sty}{\usepackage{upquote}}{}
% use microtype if available
\IfFileExists{microtype.sty}{%
\usepackage{microtype}
\UseMicrotypeSet[protrusion]{basicmath} % disable protrusion for tt fonts
}{}
\usepackage[margin=1in]{geometry}
\usepackage{hyperref}
\hypersetup{unicode=true,
            pdftitle={ilincs signature enrichment examples},
            pdfborder={0 0 0},
            breaklinks=true}
\urlstyle{same}  % don't use monospace font for urls
\usepackage{color}
\usepackage{fancyvrb}
\newcommand{\VerbBar}{|}
\newcommand{\VERB}{\Verb[commandchars=\\\{\}]}
\DefineVerbatimEnvironment{Highlighting}{Verbatim}{commandchars=\\\{\}}
% Add ',fontsize=\small' for more characters per line
\usepackage{framed}
\definecolor{shadecolor}{RGB}{248,248,248}
\newenvironment{Shaded}{\begin{snugshade}}{\end{snugshade}}
\newcommand{\KeywordTok}[1]{\textcolor[rgb]{0.13,0.29,0.53}{\textbf{#1}}}
\newcommand{\DataTypeTok}[1]{\textcolor[rgb]{0.13,0.29,0.53}{#1}}
\newcommand{\DecValTok}[1]{\textcolor[rgb]{0.00,0.00,0.81}{#1}}
\newcommand{\BaseNTok}[1]{\textcolor[rgb]{0.00,0.00,0.81}{#1}}
\newcommand{\FloatTok}[1]{\textcolor[rgb]{0.00,0.00,0.81}{#1}}
\newcommand{\ConstantTok}[1]{\textcolor[rgb]{0.00,0.00,0.00}{#1}}
\newcommand{\CharTok}[1]{\textcolor[rgb]{0.31,0.60,0.02}{#1}}
\newcommand{\SpecialCharTok}[1]{\textcolor[rgb]{0.00,0.00,0.00}{#1}}
\newcommand{\StringTok}[1]{\textcolor[rgb]{0.31,0.60,0.02}{#1}}
\newcommand{\VerbatimStringTok}[1]{\textcolor[rgb]{0.31,0.60,0.02}{#1}}
\newcommand{\SpecialStringTok}[1]{\textcolor[rgb]{0.31,0.60,0.02}{#1}}
\newcommand{\ImportTok}[1]{#1}
\newcommand{\CommentTok}[1]{\textcolor[rgb]{0.56,0.35,0.01}{\textit{#1}}}
\newcommand{\DocumentationTok}[1]{\textcolor[rgb]{0.56,0.35,0.01}{\textbf{\textit{#1}}}}
\newcommand{\AnnotationTok}[1]{\textcolor[rgb]{0.56,0.35,0.01}{\textbf{\textit{#1}}}}
\newcommand{\CommentVarTok}[1]{\textcolor[rgb]{0.56,0.35,0.01}{\textbf{\textit{#1}}}}
\newcommand{\OtherTok}[1]{\textcolor[rgb]{0.56,0.35,0.01}{#1}}
\newcommand{\FunctionTok}[1]{\textcolor[rgb]{0.00,0.00,0.00}{#1}}
\newcommand{\VariableTok}[1]{\textcolor[rgb]{0.00,0.00,0.00}{#1}}
\newcommand{\ControlFlowTok}[1]{\textcolor[rgb]{0.13,0.29,0.53}{\textbf{#1}}}
\newcommand{\OperatorTok}[1]{\textcolor[rgb]{0.81,0.36,0.00}{\textbf{#1}}}
\newcommand{\BuiltInTok}[1]{#1}
\newcommand{\ExtensionTok}[1]{#1}
\newcommand{\PreprocessorTok}[1]{\textcolor[rgb]{0.56,0.35,0.01}{\textit{#1}}}
\newcommand{\AttributeTok}[1]{\textcolor[rgb]{0.77,0.63,0.00}{#1}}
\newcommand{\RegionMarkerTok}[1]{#1}
\newcommand{\InformationTok}[1]{\textcolor[rgb]{0.56,0.35,0.01}{\textbf{\textit{#1}}}}
\newcommand{\WarningTok}[1]{\textcolor[rgb]{0.56,0.35,0.01}{\textbf{\textit{#1}}}}
\newcommand{\AlertTok}[1]{\textcolor[rgb]{0.94,0.16,0.16}{#1}}
\newcommand{\ErrorTok}[1]{\textcolor[rgb]{0.64,0.00,0.00}{\textbf{#1}}}
\newcommand{\NormalTok}[1]{#1}
\usepackage{graphicx,grffile}
\makeatletter
\def\maxwidth{\ifdim\Gin@nat@width>\linewidth\linewidth\else\Gin@nat@width\fi}
\def\maxheight{\ifdim\Gin@nat@height>\textheight\textheight\else\Gin@nat@height\fi}
\makeatother
% Scale images if necessary, so that they will not overflow the page
% margins by default, and it is still possible to overwrite the defaults
% using explicit options in \includegraphics[width, height, ...]{}
\setkeys{Gin}{width=\maxwidth,height=\maxheight,keepaspectratio}
\IfFileExists{parskip.sty}{%
\usepackage{parskip}
}{% else
\setlength{\parindent}{0pt}
\setlength{\parskip}{6pt plus 2pt minus 1pt}
}
\setlength{\emergencystretch}{3em}  % prevent overfull lines
\providecommand{\tightlist}{%
  \setlength{\itemsep}{0pt}\setlength{\parskip}{0pt}}
\setcounter{secnumdepth}{0}
% Redefines (sub)paragraphs to behave more like sections
\ifx\paragraph\undefined\else
\let\oldparagraph\paragraph
\renewcommand{\paragraph}[1]{\oldparagraph{#1}\mbox{}}
\fi
\ifx\subparagraph\undefined\else
\let\oldsubparagraph\subparagraph
\renewcommand{\subparagraph}[1]{\oldsubparagraph{#1}\mbox{}}
\fi

%%% Use protect on footnotes to avoid problems with footnotes in titles
\let\rmarkdownfootnote\footnote%
\def\footnote{\protect\rmarkdownfootnote}

%%% Change title format to be more compact
\usepackage{titling}

% Create subtitle command for use in maketitle
\newcommand{\subtitle}[1]{
  \posttitle{
    \begin{center}\large#1\end{center}
    }
}

\setlength{\droptitle}{-2em}

  \title{ilincs signature enrichment examples}
    \pretitle{\vspace{\droptitle}\centering\huge}
  \posttitle{\par}
    \author{}
    \preauthor{}\postauthor{}
    \date{}
    \predate{}\postdate{}
  

\begin{document}
\maketitle

This is an \href{http://rmarkdown.rstudio.com}{R Markdown} Notebook.
When you execute code within the notebook, the results appear beneath
the code.

Try executing this chunk by clicking the \emph{Run} button within the
chunk or by placing your cursor inside it and pressing
\emph{Ctrl+Shift+Enter}.

\begin{Shaded}
\begin{Highlighting}[]
\KeywordTok{require}\NormalTok{(httr)}
\end{Highlighting}
\end{Shaded}

\begin{verbatim}
## Loading required package: httr
\end{verbatim}

\begin{Shaded}
\begin{Highlighting}[]
\KeywordTok{require}\NormalTok{(jsonlite)}
\end{Highlighting}
\end{Shaded}

\begin{verbatim}
## Loading required package: jsonlite
\end{verbatim}

\begin{Shaded}
\begin{Highlighting}[]
\NormalTok{glist <-}\StringTok{ "LIMA1,TMPO,KHSRP,C17orf85,ZNF740,TERF2IP,PFKP,ZNF672,RPL12,PPP1R10,PDPK1,RPS6KA3,PRRC2C,NOLC1,UHRF1BP1L,CCNYL1,NFATC2IP,LARP4B,PRRC2A,WDR20,PAK2,NANS,EIF4A3,GPATCH8,MAP3K2,RPS6KA1,BRAF,PLEC,FASN,DPF2,LRWD1,ZC3HC1,DYRK1A,CDK1,RSF1,DDX54,NCOR2,CASC3,JUND,WDR26,SRRM1,HAT1,RBBP6,ABI1,NUP214,SMARCC1,MAP3K7,OCIAD1,KIF4A,CHAMP1,DHX16,UBE2O,THRAP3,ZC3H14,RBM17,FAM129B,TMSB4X,ANLN,AHNAK,MAP4,IQGAP3,FOSL2,FAM76B,ATAD2,ATRIP,TPX2,SYNRG,RBM14,BRD4,GPALPP1,SH3KBP1,MARK2,SRRM2,NUFIP2,ALS2,WAC,ATXN2L,HN1,RNF169,KIAA0930,CLTA,VPRBP,SLC38A1,ULK1,RPS6,NOC2L,HIST1H3A,H3F3A,ESR1,IGFBP2,PGR,RPS6KB1,MYC,PSAT1,GATA3,CAV1,JUN,MAPK14,CHI3L1,WWTR1,STAT3,AR,TSC2,PTCH1,MAML2,CCNB1,CNST,PTEN,BAD,ACACA,PRKCA,EGFR,COMT,CCNE1,MTOR,PRKAA1,CCNE2,YBX1,STMN1,TEP1,PCNA,TPT1,MALT1,FOXO3,MLLT10,FANCE,ETV6,COL6A1,TP53,ERBB2,CDH2,CD4,KIT,SERPINE1,PML,CCND1,FANCA,COL5A1,FGFR2,FN1,STK11,KIAA1324,CDK2,PARP1,MAP2K1,CTNNB1,EN1,IGF1R,GSN,CDH1,RAB25,FGFR1,TAZ,YAP1,BRCA2,JAZF1,MAPK8,RBM15,YWHAZ,PIK3CA,STAT6,BRCA1,CDKN1A,MYH11,ROPN1,MS4A1,TOP2A,CAPN1,CSRP1,AKR1C1,HIST1H1C,HNRNPA2B1,LAMA3,POLR2K,SLC7A2,SQLE,TPD52L2,MAPKAPK3,ALDH5A1,KLF4,KIAA0100,KIAA0355,TXNIP,SLC2A6,NOSIP,SLC29A3,PAK6,CHPT1,TRIB3,C17orf62,DENND2D,NOA1,C9orf3,TUBB"}
 
\NormalTok{myurl <-}\StringTok{ "http://www.ilincs.org/api/ilincsR/GeneListEnrichment"}
\NormalTok{res <-}\StringTok{ }\KeywordTok{POST}\NormalTok{(myurl, }\DataTypeTok{body =} \KeywordTok{list}\NormalTok{(}\DataTypeTok{geneList=}\NormalTok{glist, }\DataTypeTok{libName=}\StringTok{"LIB_6"}\NormalTok{), }\DataTypeTok{encode =} \StringTok{"form"}\NormalTok{)}
\NormalTok{output <-}\StringTok{ }\NormalTok{data.table}\OperatorTok{::}\KeywordTok{rbindlist}\NormalTok{(}\KeywordTok{content}\NormalTok{(res), }\DataTypeTok{use.names =} \OtherTok{TRUE}\NormalTok{, }\DataTypeTok{fill =} \OtherTok{TRUE}\NormalTok{)}

\KeywordTok{head}\NormalTok{(output)}
\end{Highlighting}
\end{Shaded}

\begin{verbatim}
##       tableNames zScores treatment perturbagenID cellLine  time
## 1: LINCSKD_13365  5.0229     PRDX2   CGS001-7001   HCC515  96 h
## 2: LINCSKD_13722  4.9311     SOX11   CGS001-6664   HCC515  96 h
## 3:   LINCSKD_521  4.9024     CDK16   CGS001-5127     A375  96 h
## 4: LINCSKD_21213  4.8988     ANXA7    CGS001-310     MCF7 144 h
## 5:  LINCSKD_8753  4.7703     CSRP1   CGS001-1465     HA1E  96 h
## 6: LINCSKD_25562  4.6619     VGLL4   CGS001-9686      NPC  96 h
\end{verbatim}

Add a new chunk by clicking the \emph{Insert Chunk} button on the
toolbar or by pressing \emph{Ctrl+Alt+I}.

When you save the notebook, an HTML file containing the code and output
will be saved alongside it (click the \emph{Preview} button or press
\emph{Ctrl+Shift+K} to preview the HTML file).

The preview shows you a rendered HTML copy of the contents of the
editor. Consequently, unlike \emph{Knit}, \emph{Preview} does not run
any R code chunks. Instead, the output of the chunk when it was last run
in the editor is displayed.


\end{document}
